% !TEX encoding = UTF-8 Unicode
% !TEX TS-program = LuaLaTeX



\documentclass[8pt]{beamer}


\mode<presentation> {
    % The Beamer class comes with a number of default slide themes
    % which change the colors and layouts of slides. Below this is a list
    % of all the themes, uncomment each in turn to see what they look like.

    \usetheme{default}
    %\usetheme{AnnArbor}
    %\usetheme{Antibes}
    %\usetheme{Bergen}
    %\usetheme{Berkeley}
    %\usetheme{Berlin}
    %\usetheme{Boadilla}
    %\usetheme{CambridgeUS}
    %\usetheme{Copenhagen}
    %\usetheme{Darmstadt}
    %\usetheme{Dresden}
    %\usetheme{Frankfurt}
    %\usetheme{Goettingen}
    %\usetheme{Hannover}
    %\usetheme{Ilmenau}
    %\usetheme{JuanLesPins}
    %\usetheme{Luebeck}
    %\usetheme{Madrid}
    %\usetheme{Malmoe}
    %\usetheme{Marburg}
    %\usetheme{Montpellier}
    %\usetheme{PaloAlto}
    %\usetheme{Pittsburgh}
    %\usetheme{Rochester}
    %\usetheme{Singapore}
    %\usetheme{Szeged}
    %\usetheme{Warsaw}

    % As well as themes, the Beamer class has a number of color themes
    % for any slide theme. Uncomment each of these in turn to see how it
    % changes the colors of your current slide theme.

    %\usecolortheme{albatross}
    %\usecolortheme{beaver}
    %\usecolortheme{beetle}
    \usecolortheme{crane}
    %\usecolortheme{dolphin}
    %\usecolortheme{dove}
    %\usecolortheme{fly}
    %\usecolortheme{lily}
    %\usecolortheme{orchid}
    %\usecolortheme{rose}
    %\usecolortheme{seagull}
    %\usecolortheme{seahorse}
    %\usecolortheme{whale}
    %\usecolortheme{wolverine}

    %\setbeamertemplate{footline} % To remove the footer line in all slides uncomment this line
    %\setbeamertemplate{footline}[page number] % To replace the footer line in all slides with a simple slide count uncomment this line

    %\setbeamertemplate{navigation symbols}{} % To remove the navigation symbols from the bottom of all slides uncomment this line
}

\usepackage{graphicx} % Allows including images
\usepackage{booktabs} % Allows the use of \toprule, \midrule and \bottomrule in tables



%----------------------------------------------------------------------------------------
%	DOCUMENT
%----------------------------------------------------------------------------------------

\begin{document}



    \begin{frame}
        \frametitle{Learning Model Predictive Control}
        AIM: The objective of Learning MPC is to design a reference-free iterative control strategy, able to learn from previous iterations.
        
        An infinite horizon optimal control problem can be approximated as a finite horizon optimal control problem for MPC fomrulation:
        
        \begin{equation}
            \begin{aligned}
                \min_{U_k} \quad & \sum_{i = 0}^{\infty} \left( h(x_{k}, u_{k}) \right)   \\
                \textrm{s.t.} \quad & x_{k+1} = Ax_{k} + Bu_{k} \\
                                    & x_{0} = x_{s} \\
                                    & x_{k} \in X_{c}  \\
                                    & u_{k} \in U_{c}  \\
            \end{aligned}
            \Longrightarrow 
            \begin{aligned}
                \min_{U_k} \quad & \sum_{i = 0}^{N_p-1} \left( h(x_{k}, u_{k}) + \boxed{ Q^{j-1}(x_{N_p|k}) } \right)    \\
                \textrm{s.t.} \quad & \boxed{ x_{i+1|k} = f(x_{i|k}, u_{i|k}) }, \quad i \in [0, 1, ..., N_p-1] \\
                                    & x_0 = x^{j}_{0|k}  \\
                                    & x_{i|k} \in X_{c}, \quad i \in [0, 1, ..., N_p-1] \\
                                    & u_{i|k} \in U_{c}, \quad i \in [0, 1, ..., N_p-1] \\
                                    & x_{N_p|k} \in \boxed{ SS^{j-1} }  \\
            \end{aligned}
        \end{equation}
        where:  $ \begin{array}[t]{l}
            SS^{j-1} \quad $terminal constraint (Safe Set)$  \\
            Q^{j-1} \quad $terminal cost function (Iteration Cost)$  \\
        \end{array} $ \\

        Assumption: There exists a initial feasible trajectory $x^0$, which starts from the initial state $x_0$ and converging to the final goal $x_f$.

    \end{frame}
    



    \begin{frame}
        \frametitle{Safe Set}

        Terminal constraint is needed to ensure:
        \begin{itemize}
            \item Recursive feasibility
            \item Iterative feasibility
        \end{itemize}

        It is defined the Safe Set SS at j-th iteration, as the set of all the successful trajectories performed in the first j iterations:
        \begin{equation}
            SS^j = \left\{ \bigcup_{i \in M^j} \bigcup_{t=0}^{\infty}x_{t}^{i} \right\} 
        \end{equation}
        where:  $ \begin{array}[t]{l}
            M^j \quad $set of trajectories of successful iterations$  \\
        \end{array} $ \\

        \hfill \break

        Particular case: Linear System
        For linear systems it is possible to relax the Safe Set to its convex hull, defined Convex Set:
        \begin{equation}
            CS^j = Conv(SS^j)
        \end{equation}

    \end{frame}



    \begin{frame}
        \frametitle{Iteration Cost}

        Terminal cost is needed to ensure:
        \begin{itemize}
            \item Asymptotical stability
            \item Performance improvement
        \end{itemize}

        It is defined the Iteration cost:
        \begin{equation}
            Q^j(x) = \begin{cases}
                \min_{(i,t) \in F^j(x)} \quad J_{t\rightarrow\infty}^{i}(x) \quad if \; x \in SS^j \\
                +\infty, \quad if \; x \notin SS^j
            \end{cases}
        \end{equation}
        where:  $ \begin{array}[t]{l}
            J_{t\rightarrow\infty}^{i}(x) = \sum_{i = 0}^{\infty} \left( h(x_{i|k}^{j}, u_{i|k}^{j}) \right)  \quad $terminal cost (cost-to-go)$ \\
        \end{array} $ \\

        \hfill \break

        Particular case: Linear System
        For linear systems it is possible to relax the iteration cost, to its convex function:
        \begin{equation}
            P^j(x) = \begin{cases}
                p^j(x) \quad if \; x \in SS^j \\
                +\infty, \quad if \; x \notin SS^j \\
            \end{cases}
        \end{equation}

        which is a LP:
        \begin{equation}
            \begin{aligned}
                p^j(x) = \min_{\lambda_t^j \geqslant 0} \quad & \sum_{k=0}^{j} \sum_{t=0}^{t_j} \left( \lambda_t^k J_{t\rightarrow\infty}^{i}(x) \right)   \\
                \textrm{s.t.} \quad & x_{k+1} = Ax_{k} + Bu_{k}, \quad x_{0} = x_{s} \\
                                    & x_{k} \in X_{c}  \\
                                    & u_{k} \in U_{c}  \\
            \end{aligned}
        \end{equation}

    \end{frame}


\end{document} 